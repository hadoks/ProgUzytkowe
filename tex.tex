\documentclass{article}
\usepackage[a4paper,left=3.5cm,right=2.5cm,top=2.5cm,bottom=2.5cm]{geometry}
%%\usepackage[MeX]{polski}
%%\usepackage[cp1250]{inputenc}
\usepackage{polski}
\usepackage[utf8]{inputenc}
\usepackage[pdftex]{hyperref}
\usepackage{makeidx}
\usepackage[tableposition=top]{caption}
\usepackage{algorithmic}
\usepackage{graphicx}
\usepackage{enumerate}
\usepackage{multirow}
\usepackage{amsmath} %pakiet matematyczny
\usepackage{amssymb} %pakiet dodatkowych symboli






\begin{document}
\tableofcontents
\section{tytuł}
\subsection{podtytuł}
\cite{dote:ks} mówi o \TeX

\begin{itemize}
\item punkt 1
\item punkt 2

\end{itemize}

\begin {enumerate}
\item punkt 1
\item punkt 2

\end {enumerate}

\begin {description}

\item [itemize] to środowisko do wypunktowania
\item [enumerate] to środowisko do numerowania punktów

\end{description}

\begin{thebibliography}{1}

\bibitem{dote:ks} Donald E. Knuth, \TeX Przeowdnik użytkownika, WNT, Warszawa

\end{thebibliography}

\end{document}
