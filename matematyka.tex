\documentclass{article}
\usepackage[a4paper,left=3.5cm,right=2.5cm,top=2.5cm,bottom=2.5cm]{geometry}
%%\usepackage[MeX]{polski}
%%\usepackage[cp1250]{inputenc}
\usepackage{polski}
\usepackage[utf8]{inputenc}
\usepackage[pdftex]{hyperref}
\usepackage{makeidx}
\usepackage[tableposition=top]{caption}
\usepackage{algorithmic}
\usepackage{graphicx}
\usepackage{enumerate}
\usepackage{multirow}
\usepackage{amsmath} %pakiet matematyczny
\usepackage{amssymb} %pakiet dodatkowych symboli
\begin{document}
Tu umieszczamy kod TeXa, ktory bedzie kompilowany,
\begin{displaymath}
S^{C_{i}}(a)=\frac{(\overline{C}^{a}_{i}-\widehat{C}^{a}_{i})^2}{Z_{\overline{C}^{{a}^2}_{i}}+Z_{\widehat{C}^{a^2}_{i}}}, a \in A
\end{displaymath}
\begin{equation}
\left[
\begin{array} {cccc}
a_{11} & a_{12} & \ldots & a_{1K} \\
a_{21} & a_{22} & \ldots & a_{1K} \\
\vdots & \vdots & \ddots & \vdots \\
a_{k1} & a_{K2} & \ldots & a_{KK} \\
\end{array}
\right]*
\left[
\begin{array}{c}
x_{1} \\
x_{2} \\
\vdots \\
x_K \\
\end{array}
\right]=
\left[
\begin{array}{c}
b_1 \\
b_2 \\
\vdots \\
b_k \\
\end{array}
\right]
\end{equation}
\begin{verbatim}
for(int i=0;i<10;i++)
{
cout<<"i="<<i;
}
\end{verbatim}
\begin{algorithmic}
\FOR {i=0,i,$\ldots$,10}
\item{verb+cout<<"i="'<<i;+}
\ENDFOR
\end{algorithmic}
\begin{displaymath}
\lim_{n\rightarrow\infty} \sum^{n}_{k=1}\frac{1}{k2}=\frac{\pi^2}{6}
\end{displaymath}
\begin{displaymath}
[x]_A = \{y \in U : a(x) = a(y), \forall a \in A\}, where\ the\ central\ object\ x \in U
\end{displaymath}
\begin{displaymath}
cos(20)=cos^2 0 - sin^2 0
\end{displaymath}
\end{document}